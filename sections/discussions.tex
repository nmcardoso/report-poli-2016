\section*{Análises}
A possibilidade de fazer observações em um local mesmo que o smartphone não esteja conectado à internet é um ponto forte do aplicativo e garante que as observações posam ser feitas de qualquer lugar. Os dados ficam armazenados na memória interna do aparelho e, quando conectado à rede, os dados são enviados para o servidor.

A partir das informações geográficas de latitude e longitude obtidas pelo próprio sensor do smartphone ou fornecidas pelo voluntário, assim que uma observação é enviada para o servidor, é adicionado um ponto no mapa representando a localização do espécime observado. Clicando sobre o ponto é possível obter informações detalhadas da observação. Além disso, para aprimorar a análise dos dados, é possível aplicar combinações de filtros que restringem a apresentação das observações no mapa. Essas restrições podem ser de acordo com o táxon, data ou local da observação. Assim, é possível observar a ocorrência de uma espécie em uma determinada época do ano ou sua presença em uma determinada região. Diversas combinações podem ser feitas de acordo com a necessidade da análise.

A estrutura de rede social cria vínculos entre os voluntários e pesquisadores e propicia o engajamento de mais voluntários à observação da natureza. Além disso, profissionais podem revisar a classificação do espécime feita pelos voluntários e sinalizar algo que esteja identificado incorretamente.
• Uso de internet no 'auto-complete' e possível implementação offline.

• Auto-complete otimizado para pesquisas utilizando nome popular em português

• Ventagens da implementação dos 'badges' ('Crowdsourcing the identification of organisms: A case-study of iSpot.').