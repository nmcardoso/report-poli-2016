\section{Resultados}
Uma nova versão do iNaturalist, traduzida para o português e com resursos adicionais, foi implementada. Houve modificação em todos os principais componentes do sistema: website (front-end), servidor (back-end) e aplicativo Android.

Quando o desenvolvimento adquiriu um nível de estabilidade adequadod, o sistema foi implantado em uma máquina virtual do LAA com um IP fixo e com acesso à rede da USP, sendo, assim, disponibilizado para acesso de qualquer dispositivo pela internet. Esta infraestrutura implementada está ilustrada na figura \ref{fig:arch}.

O aplicativo Android não foi lançado na plataforma de distribuição de aplicativos oficial do Google, mas o arquivo de instalação foi disponibilizado no site para download e instalação manual.
