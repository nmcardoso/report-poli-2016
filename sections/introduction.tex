\section*{Introdução}
\emph{Citizen science} refere-se ao engajamento de voluntários amadores na investigação científica levantando questionamentos, coletando dados ou interpretando resultados. Em geral, projetos \emph{citizen science} geralmente incluem uma parceria entre amadores e pesquisadores. \cite{sivertown2009} Num contexto de biodiversidade, os voluntários podem contribuir fazendo identificação de espécimes. Assim, é possível mapear e analisar as ocorrências de espécies em uma determinada região, identificando espécies exóticas ou em extinção, por exemplo. \cite{miller2012} Com o auxílio da tecnologia, é possível desenvolver plataformas que agilizem a coleta e o processamento dos dados. \cite{bonney2014}

Uma plataforma online com estrutura de rede social que integra voluntários e pesquisadores é decisivo para o engajamento entre as pessoas envolvidas e, consequentemente, a popularização do projeto. Os voluntários publicam suas observações e, ao mesmo tempo, podem aprender as pesquisas desenvolvidas. Esta troca de conhecimento entre profissionais e amadores é uma das principais características de projetos \emph{citizen science}. \cite{sivertown2009}

Como os dados coletados por voluntários podem chegar a um grau de confiabilidade muito próximo ao de profissionais \cite{gollan2012, vanstrien2013}, uma base de dados com observações georreferenciadas de espécimes coletadas por voluntários é uma riquíssima fonte de dados para projetos de pesquisa da área. Mais do que isso, uma plataforma que conecta pesquisadores e voluntários tem grande potencial de intercâmbio de informações. 
