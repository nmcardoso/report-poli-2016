\section*{Conclusão}

Esta é uma importante ferramenta tanto para ciência quanto para a agricultura, pois guardar informações de espécies presentes num certo momento em um determinado local é uma rica fonte de dados para basear estudos em biodiversidade e tomadas e decisões numa área de cultivo de alimentos. A disposição das observações num mapa e a restrição dos dados por filtros deixa simples e intuitiva a apresentação das ocorrências de espécies numa determinada área ou período, o que agiliza e facilita a análise dos dados.
  
Aproveitando o potencial de monitoramento de espécies desta plataforma, uma possível aplicação é usá-la num contexto de saúde humana. Ou seja, focar no monitoramento de espécies vetores de doenças para o ser humano ou de possíveis espécies que estejam infectadas por uma nova doença.
    
• Estrutura de rede social e 'badges' na formação de uma 'reputação virtual' e influência na qualidade de dados.
    
• Vantagens de uma aplicação no idioma nativo do usuário. Tanto interface quanto dados (árvore taxonômica para pesquisas utilizando nomes populares pt-br)